\chapter{Post Processing}
Any software related to what that happens after the trace, is stored under 'memtrace-pass/post-processing'.
The trace files can be parsed with the 'extract-subset.py' python3 script. From the original full trace, the communication 
subset is extracted. Three data structures are generated by 'extract-subset.py'. All structures are stored using pickle.
The data structures are of type 'AutoDict()', which is defined in 'TraceInc.py'
\begin{enumerate}
	\item Load/Store volumes, both total and during communication, for Kernels, CTAs and SMs. Separated by kernel and superstep.
	\begin{lstlisting}[style=C]
"KCV" : { // Kernel Communication Volume
	<kernel>: {
		<superstep>: {
			"Load" : <int>
			"Store": <int>
		}
	}
}
"KDV" : { // Total Kernel Data Volume
	<kernel>: {
		<superstep>: {
			"Load" : <int>
			"Store": <int>
		}
	}
}
"CCV" : { // CTA Communication Volume
	<kernel>: {
		<superstep>: {
			<CTA> : {
				"Load" : <int>
				"Store": <int>
			}
		}
	}
}

"CDV" : { // Total CTA Data Volume
	<kernel>: {
		<superstep>: {
			<CTA> : {
				"Load" : <int>
				"Store": <int>
			}
		}
	}
}
"SCV" : { // SM Communication Volume
	<kernel>: {
		<superstep>: {
			<SM> : {
				"Load" : <int>
				"Store": <int>
			}
		}
	}
}

"SDV" : { // Total SM Data Volume
	<kernel>: {
		<superstep>: {
			<SM> : {
				"Load" : <int>
				"Store": <int>
			}
		}
	}
}

	\end{lstlisting}
	\item Map of transfers, ordered by kernel, CTA and superstep.
	\begin{lstlisting}[style=C]
<source-kernels> : {
	<source-cta> : {
		<source-superstep> : {
			<recv-kernel>: {
				<recv-cta>: {
					<recv-superstep>: {
						'Size' : <int>
						'cnt'  : <int>
					}
				}
			}
		}
	}
}
	\end{lstlisting}
	\item All addresses involved in communication, with all operations performed on this address, in superstep order.
	\begin{lstlisting}[style=C]
{
	<address> : [
		{ // record
			"kernel": <kernel>
			"it"    : <superstep>
			"cta"	: <CTA-ID, xyz order>
			"addr"  : <address>
			"smid"  : <SM id>
			"size"  : <operation size in bytes>
			"type"  : <type of MOp>
		}
	]
}
	\end{lstlisting}
\end{enumerate}