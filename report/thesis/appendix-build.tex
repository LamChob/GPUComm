\chapter{Instrumentation}
The source-code for this work will be available in the Computer Engineering Group Git repository (\verb|https://github.com/UniHD-CEG|). It includes the Clang and LLVM extensions, as well as
the utilities to analyse traces.
Instrumenting an application at compile time happens in two steps. The first step is the instrumenting all 
source file containing:
\begin{itemize}
	\item Kernel Calls
	\item Kernel and \verb|__device__| function definitions
	\item Kernel \verb|__device__| functoin declarations
\end{itemize}

Each file needs to be parsed separately. For each file, a new file with the prefix 'augmented-' is created. This is then used for
the further build process. Compile units bear some problems with kernels defined and declared in \verb|.cu| files, which are then included
in other files. The easiest workaround is to copy the kernel into the file including it, and then only use the including file for instrumentation and build.

The following snipped is an example how to augment a kernel file.
\begin{lstlisting}[style=C]
clang++ -Xclang -load -Xclang "clang-plugin/libMemtrace-AA.so"
	 -Xclang -plugin -Xclang cuda-aug -Xclang -plugin-arg-cuda-aug 
	 -Xclang -f -Xclang -plugin-arg-cuda-aug -Xclang ./augmented-kernel.cu 
	 -I$CUDASYS/samples/common/inc -I. -I../utils
     --cuda-path=$CUDASYS -std=c++11 -E application/kernel.cu 
\end{lstlisting}

Next, the host and device utils are compiled, for later linking with the application.
\begin{lstlisting}[style=C]
// Host
clang++ -c -L$CUDASYS/lib64 --cuda-path=$CUDASYS 
 -I$CUDASYS/samples/common/inc -O1 --cuda-gpu-arch=sm_30
  -I$CUDASYS/include  -o hutils.o --std=c++11 -I. -I../utils
 ../utils/TraceUtils.cpp
// Device
clang++ -c -L$CUDASYS/lib64 --cuda-path=$CUDASYS 
 -I$CUDASYS/samples/common/inc -O1 --cuda-gpu-arch=sm_30
 -I$CUDASYS/include -o dutils.o --std=c++11 -I. -I../utils
 ../utils/DeviceUtils.cu
\end{lstlisting}
Next, all the augmented kernel files are compiled at once.
\begin{lstlisting}[style=C]
clang++ -c -Xclang -load -Xclang $LLVMPLUGIN  --cuda-path=$CUDASYS
 -I$CUDASYS/samples/common/inc --cuda-gpu-arch=sm_30 -L$CUDASYS/lib64  -O1
 -lcudart_static -m64  --std=c++11 -I. -I../utils -I<app-includes>
  ./augmented-kernel.cu ./augmented-kernel2.cu
\end{lstlisting}
Finally, all compiled files are linked together.
\begin{lstlisting}[style=C]
clang++ --cuda-path=$CUDASYS -I$CUDASYS/samples/common/inc --cuda-gpu-arch=sm_30 -L$CUDASYS/lib64  -O1
 -lcudart -ldl -lrt -L. -m64 --std=c++11 -I. -I../utils -I<app-includes>
 -o application dutils.o hutils.o augmented-kernel.o augmented-kernel2.o
\end{lstlisting}

During the execution one or more files are generated in the \verb|/tmp| folder. The files are named \verb|MemTrace-pipe-<n>|, one for
each stream.
