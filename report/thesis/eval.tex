\chapter{Evaluation and Analysis of GPU Communication}\label{eval}
This chapter examines anylsis results of the traces generated with the technique described in chapter \ref{chap:impl}.
The analysis will focus on aggregated statistics, rather than averaging. 

\section{Evaluated Applications}
	Set of Functions that can display a communicating behavior, tend to be iterative kernels
\begin{itemize}
	\item Histogram (CUDA Suite)
	\item Stencil (Rodinia 2D, 3D)
	\item NBody (CEG)
	\item BFS (Rodinia)
	\item Pathfinder (Rodinia)
\end{itemize}
%\section{Analysis Parameters}
%\subsection{Communication Classes}
%\begin{itemize}
%	\item P2P: One Writer, one Reader
%	\item Scatter: One Writer, Multiple Readers
%	\item Gather: Multiple Writes, one Reader
%	\item Synchronization: Atomic access to one Address by multiple CTAs
%\end{itemize}
%\subsection{Communication Metrics}
%\begin{itemize}
%	\item Write-to-Read ratio
%	\item Transfer Size Histogram: Data Transferred in one Communication between two CTAs
%	\item Transfer Density: Number of Communications(one Write->Read relation) between CTAs in whole application
%	\item Volume Density: Number of bytes transferred in whole application
%	\item Sparcity: Stride in data Communicated via Scatter/Gather
%\end{itemize}
\section{Level I Analysis}
\subsection{Density Plots}
\subsection{Communication Fraction}
\subsection{CTA In/Out-Degree}
\subsection{Msg-size Histogram/CDF}
\section{Level II Analysis}
\subsection{Kernel communication Evolution}
\subsection{Bisection Bandwidth}
\section{Level III Analysis}
\subsection{Philandering}
\subsection{ComPartner Ratio Evolution}